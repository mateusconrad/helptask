\chapter*{Introdução} \label{chap:intro}

A tecnologia da informação pode ser considerada como um conjunto de tecnologias para processar e armazenar dados com o uso de hardware, software, comunicação e pessoas. A TI, na verdade, procura facilitar processos através da eliminação de rotinas que possam ser muito repetitivas e também reduzir a quantidade de erros operacionais de usuários. 

Assim como qualquer área de negócio, a TI tem a necessidade de ser gerenciada, então surge a área de gerenciamento de serviços de TI. Essa gestão procura usar soluções de software ou até mesmo mudanças de metodologias internas para atingir mais agilidade nos processos. Também pode ser colocada como a prática estruturada de se indicar as formas mais adequadas de atender as necessidades tecnológicas uma organização. 

Para realizar isso o gerenciamento de serviços de TI deve ter a avaliação dos recursos tecnológicos disponíveis, determinar quais as restrições de recursos, implementar as mudanças e otimizações no serviço de TI para procurar remover as restrições, gerenciar as melhorias de forma integrada, fazer o monitoramento dos resultados, procurar avaliar a necessidade de novas otimizações e atualizações dos serviços de TI, atender chamados, quando necessário, e, por fim, procurar diminuir ocorrências, principalmente as ocorrências com repetição frequente.

Os dispositivos móveis traçaram um longo caminho nos últimos anos. Por um tempo, os  celulares eram apenas usados com o intuito fazer ligações. Todavia, a medida que esses dispositivos continuavam evoluindo, a capacidade de enviar e receber mensagens de texto e multimídia, criar agendas e salvar contatos tornou-se prontamente disponível. Atualmente os dispositivos móveis são usados em larga escala e são capazes de atender a diversos outros propósitos.

O uso de um software, como um app mobile devido a popularidade dos smartphones, direcionado a gestão de ti permitira ter sempre em mãos a capacidade de controlar os serviços de TI da empresa. Portanto, surge a necessidade de desenvolver um app para criar, monitorar e gerenciar chamados de suporte interno de uma organização. Tendo isso em vista, é necessário procurar embasamento referente a metodologias e boas praticas de gerenciamento de ti.

Um dos primeiros conceitos a se observar ao se discutir sobre gerenciamento de TI é ITIL. O ITIL é uma biblioteca que reúne um conjunto de boas praticas para gerenciamento de TI, que nesse escopo trata sobre a central de serviços, gerenciamento de incidentes, de problemas, de configurações e de mudanças.

No que diz respeito a desenvolvimento de apps mobile, é possível encontrar diversas linguagens de programação e frameworks pra auxiliar no processo, então deve-se optar pela alternativa que leva mais em conta em termos de produtividade, qualidade e inovação. 

O uso do Flutter pode se adequar em relação a produtividade, pois com apenas um código é capaz de gerar aplicativos com performance nativa para smartphones Android e Ios. Qualidade de desenvolvimento pode ser algo relativo, sendo necessária uma análise de experiência de usuário para ser possível dizer o que tem mais qualidade.

Além do desenvolvimento da interface e funcionamento do app, é necessário um meio de controlar os dados gerados através do app. Em consequência disso, pode-se buscar inovar no que diz respeito a infraestrutura do próprio aplicativo, ou seja, usar de softwares como serviço para o back-end em \textit{cloud} para fazer \textit{storage} de arquivos, estruturação de dados e funções personalizadas para o próprio aplicativo.