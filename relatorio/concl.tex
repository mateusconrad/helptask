\chapter*{Conclusão} \label{chap:concl}


Partindo, então do tema do trabalho, o qual tratava do desenvolvimento de um app mobile para chamados de suporte de uma empresa, podem se apresentar as conclusões sobre a análise realizada e o aplicativo desenvolvido.

Primeiramente, retomando os  objetivos estipulados neste trabalho, os quais traziam como objetivo geral desenvolver um aplicativo para padronizar o processo de abertura de chamados de suporte internos de uma empresa de TI pode-se verificar que este foi parcialmente cumprido, pois a aplicação foi desenvolvida e funciona, por outro lado não sofreu um período de testes de uso ou aceitação dos próprios usuários e atendentes.

Em sequência, como objetivos específicos foram propostos a realização da análise e desenvolvimento de um app para chamados de suporte, utilização de banco de dados em tempo real para gerenciamento dos dados, realizar os testes de software no aplicativo e disponibilizar o arquivo para instalação do app. Foram atingidos e concluídos os três primeiros, não houve a realização dos testes de software devido a limitações de desenvolvimento do grupo, bem como a falta de tempo para a realização e documentação dos mesmos. 

O app também não foi disponibilizado ou hospedado em lojas de aplicativos pois há custos para isso que não foram previstos e também não se viu a necessidade, pois o código do aplicativo já é aberto.

A partir do problema levantado na fase de análise, o qual questionava como um aplicativo pode padronizar e agilizar o processo de abertura de chamados de suporte internos de uma empresa, foram levantadas duas hipóteses para responder ao problema.

A primeira hipótese afirmava que O aplicativo desenvolvido pode definir uma padronização para os chamados de suporte interno na empresa. A primeira hipótese, pôde ser corroborada apesar da implantação da aplicação não ter ocorrido, pois pode se observar que a metodologia empregada é capaz de padronizar a maneira de realizar e requisitar os chamados de suporte internamente na empresa, conforme a apresentação do software presente na seção 4 do capítulo 3. O último passo efetivo para este ponto é a própria empresa tratar o app como o único canal de comunicação entre usuário e suporte. Os testes foram somente realizados pelos pesquisadores na medida em que a construção do app evoluiu. Quanto ao quarto objetivo, não pode ser atingido, pois o app foi disponibilizado somente na forma de código-fonte, ao contraponto da proposta de fornecer o arquivo de instalação.

Já a segunda hipótese afirmava que o aplicativo desenvolvido permite visualizar estatísticas de chamados realizados". Essa hipótese não pode ser corroborada nem refutada, uma vez que não houve a implementação necessária, ou seja, não houve a aplicação de gráficos em relação as informações dos próprios chamados, não houve tempo hábil para a implementação dos mesmos, como também não houve a definição de um escopo de quais informações esses gráficos iriam exibir em tela.

A análise foi realizada tendo como base a Abase Sistemas, como ponto mais específico, em seu setor responsável pela infraestrutura do ambiente, formulando assim uma aplicação que fosse capaz de gerir os chamados de suporte.

A partir dos pontos de melhoria ou que ficaram fracos no desenvolvimento do aplicativo, podem ser elencadas algumas propostas futuras, sendo essas: a disponibilização do aplicativo nas lojas \textit{App Store} e \textit{Play Store}, implementar a geração de gráficos e filtros baseados em ordenamento de data, título, prioridade e tipo de chamado, a realização da implementação dos gráficos dos chamados, e também a implementação de de um chat dentro da aplicação, assim permitindo, além da abertura do chamado, uma breve comunicação entre atendente e usuário para evitar erros.

Por meio do desenvolvimento da aplicação proposta, a qual propunha a padronização dos chamados de TI através de um app mobile para sistemas operacionais Android, pode ser observada a necessidade de um fluxo de processos bem estruturado para que a eficiência da equipe de TI cumprisse adequadamente seu papel.

O desenvolvimento da presente pesquisa foi capaz agregar novos conhecimentos aos pesquisadores, tanto no que diz respeito aos conteúdos da disciplina de Linguagem de Programação III e Análise e Estratégias Sistemas, quanto na construção de conhecimentos acerca da área de suporte das empresas, portanto, considera-se que foi de grande valia o aprendizado em relação a duas áreas da tecnologia que apesar de serem diferentes uma da outra, puderam agregar valor aos acadêmicos unindo conceitos de desenvolvimento mobile com gerenciamento de TI.